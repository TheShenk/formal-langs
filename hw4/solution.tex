\documentclass[12pt,a4paper]{article}%
\usepackage{amsthm}
\usepackage{amsmath}%
\usepackage{amsfonts}%
\usepackage{amssymb}%
\usepackage{graphicx}
\usepackage[T2A]{fontenc}
\usepackage[utf8]{inputenc}
\usepackage[english,russian]{babel}
\usepackage[linguistics]{forest}
%-------------------------------------------
\usepackage{ulem}
%-------------------------------------------
\setlength{\textwidth}{7.0in}
\setlength{\oddsidemargin}{-0.35in}
\setlength{\topmargin}{-0.5in}
\setlength{\textheight}{9.0in}
\setlength{\parindent}{0.3in}

\newtheorem{theorem}{Theorem}
\newtheorem{task}[theorem]{Задача}
\addto\captionsrussian{\renewcommand*{\proofname}{Решение}}


\newcommand{\abovemath}[2]{\ensuremath{\stackrel{\text{#1}}{#2}}}
\newcommand{\aboveeq}[1]{\abovemath{#1}{=}}
\newcommand\bydef{\aboveeq{def}}

\graphicspath{{./}}
\DeclareGraphicsExtensions{.pdf,.png,.jpg}
\begin{document}


\begin{flushright}
\textbf{Удод Максим \\
\today}
\end{flushright}

\begin{center}
\textbf{Формальные языки\\
HW04} \\
\end{center}

\task {Построить LL(1) таблицу по грамматике списков
    $$
    \begin{array}{cccccccccc}
       &S & \rightarrow & L S' \\
       &S'& \rightarrow & ; L S' & \mid & \varepsilon \\
       &L & \rightarrow &  [S] & \mid & a
    \end{array}
    $$
}
\proof {

    \begin{center}
    \begin{tabular}{ l || c | c || c | c | c | c | c }
      N  & FIRST            & FOLLOW  & [                     & ] & a                     & ; & $ \$ $                       \\ \hline
      S  & [, a             & $\$$, ] & $S \rightarrow L S'$  &   & $S \rightarrow L S'$  &   &                              \\
      S' & ;, $\varepsilon$ & $\$$, ] & & $S' \rightarrow \varepsilon$ &   & $S' \rightarrow ; L S'$ & $S' \rightarrow \varepsilon$ \\
      L  & [, a             & ;       & $L \rightarrow [S]$            & & $L \rightarrow a$ &                         &                              \\

    \end{tabular}
    \end{center}

}

\task {Осуществить синтаксический анализ алгоритмом LL(1) для 2 списков, содержащих не меньше 7 терминалов: один список должно быть корректным, другой --- нет. Для корректного списка приведите дерево вывода. Используйте грамматику, полученную в прошлом задании.}

\proof {
Строка a;[a;a]$\$$

Стэк: \xout{$\$$} \xout{S} \xout{S'} \xout{L} \xout{a} \xout{S'} \xout{L} \xout{;} \xout{]} \xout{S} \xout{[} \xout{S'} \xout{L} \xout{a} \xout{S'} \xout{L} \xout{;} \xout{a}

Дерево вывода:

\begin{center}
  \begin{forest}
  	[S
  		[L [a]]
  		[S' [
  				[;]
  				[L
  					[{[}]
  					[S
  						[L [a]]
  						[S'
  							[;]
  							[L [a]]
  							[S' [$\varepsilon$]]
  						]
  					]
  					[{]}]
  				]
  				[S' [$\varepsilon$]]
  			]
  		]
  	]
  \end{forest}
\end{center}

Строка a;[a;a;]$\$$

Стэк: $\$$ \xout{S} \xout{S'} \xout{L} \xout{a} S' \xout{L} \xout{;} ] \xout{S} \xout{[} \xout{S'} \xout{L} \xout{a} \xout{S'} \xout{L} \xout{;} \xout{a} S' L \xout{;}

Так как ячейка (L, ]) - пуста, то строка некорректна
}

\task {
    Реализовать парсер при помощи antlr4. За основу взять парсер для языка L, добавить в язык массивы. Массивы можно присваивать переменным. Также можно брать элемент массива по индексу и использовать в выражениях. Элементом массива может быть любое корректное выражение. Конкретный синтаксис для массивов придумайте сами.
}

\proof {
Ссылка на репозиторий: https://github.com/TheShenk/formal-langs

Массивы имеют следующий синтаксис: [<выражение1>, <выражение2>, <выражение3>]

Корректные примеры:\\
$[[1, 2, 3], [4, 5, 6], [7, 8, 9]]$\\
$[1+2, 3-4, 5*6, 7/8, 9 \wedge 10, 11 == 12]$\\
$[[[a[i]]]]$\\
$[[]]$\\

Некорректные примеры:\\
$[]]$\\
$[[3]$\\
$[a[i]$\\
$[3; 4]$\\

}

\end{document}